\chapter{INTRODUÇÃO}
\label{chap:introducao}

A introdução deve conter a delimitação do tema, o problema, a justificativa e o
objetivo do projeto, que podem vir em subseções separadas ou não.
É muito importante ressaltar que a delimitação do tema requer clareza a respeito do
campo de conhecimento a que pertence o assunto. O problema é o objeto de pesquisa ou de
estudo. Optou-se, neste exemplo, em separar em subseções a justificativa e o(s) objetivo(s).\\
No caso de projeto de pesquisa, que esteja vinculado a um grupo de pesquisa
institucional, neste item é necessário acrescentar a denominação do grupo, que esteja
devidamente certificado pela Unifra, e a denominação da linha de pesquisa a que pertence o
projeto

\section{JUSTIFICATIVA}
\label{sec:justificativa}
Na justificativa mencionam-se a relevância científica do trabalho, a contribuição da
pesquisa e que benefício poderá trazer à comunidade ou à sociedade. Ainda devem estar claros
o motivo da escolha do tema e as possibilidades de realização da pesquisa.
\section{OBJETIVOS}
\label{sec:objetivos}
A definição dos objetivos determina o que se quer atingir com a realização do
trabalho de pesquisa. Objetivo é sinônimo de meta, fim.
Uma sugestão interessante, na redação dos objetivos, é utilizar, no início das
sentenças, o verbo no infinitivo, tais como: esclarecer tal coisa, definir tal assunto, procurar
aquilo, permitir algo, demonstrar alguma coisa, entre outros.
Alguns autores separam os objetivos em objetivo geral e objetivos específicos, mas
não há regra a ser cumprida quanto a isso. Caso se opte em separá-los, tem-se:
\subsection{OBJETIVOS}
\label{subsec:objetivogeral}
O objetivo geral vincula-se à própria significação geral do tema proposto pelo
projeto, ou seja, significa traçar as principais metas que norteiam a pesquisa.
\subsection{Objetivo específico}
\label{subsec:objetivoespecifico}
Descrever aqui o(s) propósito(s) específico(s) para atingir um ponto de vista do tema,
um ângulo a ser pesquisado, permitindo atingir o objetivo geral. Aconselha-se, na redação
desta seção, não ser prolixo.