\chapter{REFERENCIAL TEÓRICO}
\label{chap:referencial_teorico}

O referencial teórico do assunto deve estar diretamente relacionado à pesquisa.
Requer um levantamento bibliográfico cuidadoso, que deve ser organizado numa seqüência lógica e garantir a fonte (autor, obra, data).\\
A seguir são exemplificadas citações com mais de três linhas e com menos de três
linhas.\\

No que concerne à Política Nacional de Saúde do Idoso, conforme \citeonline[p.24]{gordilho2000desafios} , destacam-se:
\begin{citacao}
a promoção do envelhecimento saudável, a manutenção e a melhoria ao máximo
possível da capacidade funcional dos idosos, a prevenção de doenças, a recuperação
da saúde daqueles que adoecem e a reabilitação daqueles que venham a ter a sua
capacidade funcional restringida.
\end{citacao}
Entre os fatores para a efetiva longevidade do ser humano um tem sido apontado, em
estudos, com muita força: a nutrição. “Várias mudanças decorrentes do processo de
envelhecimento podem ser atenuadas com uma alimentação adequada e balanceada nos
aspectos dietético e nutritivo” \cite[p.31]{salgado2002nutriccao}\\

\section{ EXEMPLO DE SEÇÃO (SEÇÃO SECUNDÁRIA) }
Aqui será alguma coisa...

\label{sec:issoeumasubsecao}

alguns exemplos de citação:\\
\cite{berquo1980fatores}\\
\cite{santos1980dinamica}\\
\cite{NBR6023:2002}\\
\cite{NBR14724:2005}\\
\cite{NBR10520:2002}\\
\cite{lessa2004manual}\\
\cite{rey2000planejar}\\
\cite{rajagopalan2003identidade}\\
\cite{flemming1999calculo}\\
\cite{gonccalves2}\\
\cite{salgado2002nutriccao}