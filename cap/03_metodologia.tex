\chapter{METODOLOGIA}
\label{chap:metodologia}
A metodologia deve ser escrita com uma linguagem precisa e técnica, seguindo uma
seqüência cronológica. Representa a descrição formal dos métodos e técnicas a serem
utilizados na pesquisa. Devem constar os métodos de abordagem e de procedimentos, os
instrumentos de coleta de dados (questionário, entrevista, formulários, entre outros), a
delimitação do universo da pesquisa, a delimitação e a seleção da amostra e do tempo
previsto, a equipe de pesquisadores e a divisão do trabalho, as formas de tabulação e o
tratamento dos dados, ou seja, a apresentação de tudo aquilo que será utilizado no trabalho de
pesquisa.\\
Em resumo, a metodologia é a explicação minuciosa, detalhada, rigorosa e exata de
toda ação desenvolvida no método (caminho) do trabalho de pesquisa.
Se a pesquisa envolver seres humanos deve atender à Resolução 196/96, do Conselho
Nacional da Saúde, e ter a aprovação do Comitê de Ética em Pesquisa (Cepe) da Unifra ou de
outra instituição.\\
\section{Notas de rodapé }
As notas de rodapé são detalhadas pela NBR 14724:2011 na seção 5.2.1\footnote{Asnotas devem ser digitadas ou datilografadas dentro das margens, ficando
separadas do texto por um espaço simples de entre as linhas e por filete de 5
cm, a partir da margem esquerda. Devem ser alinhadas, a partir da segunda linha
da mesma nota, abaixo da primeira letra da primeira palavra, de forma a destacar
o expoente, sem espaço entre elas e com fonte menor
\citeonline[5.2.1]{NBR14724:2011}.}\footnote{Caso uma série de notas sejam
criadas sequencialmente, o \abnTeX\ instrui o \LaTeX\ para que uma vírgula seja
colocada após cada número do expoente que indica a nota de rodapé no corpo do
texto.}\footnote{Verifique se os números do expoente possuem uma vírgula para
dividi-los no corpo do texto.}. 